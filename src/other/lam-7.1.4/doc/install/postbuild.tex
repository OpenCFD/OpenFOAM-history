% -*- latex -*-
%
% Copyright (c) 2001-2004 The Trustees of Indiana University.  
%                         All rights reserved.
% Copyright (c) 1998-2001 University of Notre Dame. 
%                         All rights reserved.
% Copyright (c) 1994-1998 The Ohio State University.  
%                         All rights reserved.
% 
% This file is part of the LAM/MPI software package.  For license
% information, see the LICENSE file in the top level directory of the
% LAM/MPI source distribution.
%
% $Id: postbuild.tex,v 1.11 2003/09/21 18:36:52 jsquyres Exp $
%

\chapter{After the Build}
\label{sec:postbuild}

After the build process, LAM/MPI should be built and installed.
Building the examples shows that the wrapper compilers provided by
LAM/MPI work properly.  However, there is still much of LAM that
should be tested before it is put into production use.  This chapter
details the basic testing we recommend be performed on each
installation.

\section{Sanity Check}

With LAM/MPI 7.1.x, it is possible to reconfigure many portions of LAM
that were previously relatively static.  Therefore, the \icmd{laminfo}
command has been added to allow users to determine how a particular
version of LAM is configured.  System administrators may wish to use
the \cmd{laminfo} as a sanity check when installing LAM.  An example
of the output from \cmd{laminfo} is shown below.

\lstset{style=lam-laminfo}
\begin{lstlisting}
shell$ laminfo
             LAM/MPI: 7.1.X
              Prefix: /usr/local
        Architecture: powerpc-apple-darwin7.4.0
       Configured by: jsquyres
       Configured on: Fri Jul  9 08:58:12 EST 2004
      Configure host: macosx.example.org
          C bindings: yes
        C++ bindings: no
    Fortran bindings: no
          C compiler: gcc
        C++ compiler: g++
    Fortran compiler: false
     Fortran symbols: none
         C profiling: no
       C++ profiling: no
   Fortran profiling: no
      C++ exceptions: no
      Thread support: yes
       ROMIO support: no
        IMPI support: no
       Debug support: yes
        Purify clean: yes
            SSI boot: globus (API v1.1, Module v0.5)
            SSI boot: rsh (API v1.1, Module v1.0)
            SSI boot: slurm (API v1.1, Module v1.0)
            SSI coll: lam_basic (API v1.1, Module v7.1)
            SSI coll: smp (API v1.1, Module v1.1)
             SSI rpi: crtcp (API v1.1, Module v1.0.1)
             SSI rpi: lamd (API v1.0, Module v7.0)
             SSI rpi: tcp (API v1.0, Module v7.0)
             SSI rpi: sysv (API v1.0, Module v7.0)
             SSI rpi: usysv (API v1.0, Module v7.0)
\end{lstlisting}
% stupid emacs $

The \icmd{laminfo} command includes the option \cmdarg{-parsable},
which will format the output of \cmd{laminfo} in a script friendly
way.  This has proven useful when automating sanity checks after a
build.


%%%%%%%%%%%%%%%%%%%%%%%%%%%%%%%%%%%%%%%%%%%%%%%%%%%%%%%%%%%%%%%%%%%%%%%%%%%
%%%%%%%%%%%%%%%%%%%%%%%%%%%%%%%%%%%%%%%%%%%%%%%%%%%%%%%%%%%%%%%%%%%%%%%%%%%

\section{Separating LAM and MPI TCP Traffic}
\label{sec:postbuild:mpi_hostmap}
\index{multiple networks|see {\file{lam-hostmap.txt}}}
\index{separating TCP traffic|see {\file{lam-hostmap.txt}}}
\index{lam-hostmap.txt file@\file{lam-hostmap.txt} file|(}

\changebegin{7.1}

LAM uses two distinct types of communication: its out-of-band
communications (e.g., process control, file transfer, and I/O
redirection) and MPI ``in-band'' communication.  When using the
\rpi{tcp} \kind{rpi} module, it is possible to separate these two
forms of communications onto separate networks.  

Such functionality can be useful in environments where multiple TCP
networks exist.  For example, a common cluster configuration is to
have two networks:

\begin{enumerate}
\item A Fast Ethernet network for commodity file transfers, remote
  login, and other interactive services.

\item A Gigabit Ethernet network exclusively reserved for high-speed
  data transfers (e.g., MPI communications).
\end{enumerate}

In such environments, it is desirable to use the Fast Ethernet for
LAM's out-of-band traffic, and the Gigabit Ethernet network for MPI
traffic.  By default, LAM uses one network for all of its traffic.
LAM's out-of-band communication is actually fairly sparse and
typically does not have a noticable impact on MPI communications.
However, a subtle issue arises when using LAM's \kind{boot} SSI
modules in various batch environments such as PBS (Torque) and SLURM.

These batch schedulers are typically configured to operate on the
``slow'' network in order to keep their traffic from interfering with
the high-speed traffic on the ``fast'' network.  The \boot{pbs} and
\boot{slurm} modules automatically obtain their boot schemas from
their respective batch systems, which, in these cases, will result in
a list of ``slow'' network interfaces.  Although the automatic
determination of the boot schema is a nice feature, having LAM's MPI
traffic flow across the ``slow'' network is clearly not desirable.

LAM provides a feature to map from one TCP network to another.
Specifically, LAM can map from the set of IP names/addresses specified
in the boot schema to a second set of IP names/addresses that will be
used for MPI communication.  Moreover, a default mapping can be
provided by the system administrator so that users are unaware of this
functionality.

The file \file{\$sysconf/lam-hostmap.txt} (which is usually
\file{\$prefix/etc/lam-hostmap.txt}) is the location of the default
map.  It is a simple text file that lists origin host names/addresses,
one per line, and their correspoding remapped host name/address for
use with MPI communications.  For example:

\lstset{style=lam-cmdline}
\begin{lstlisting}
node1.slow.example.com mpi=node1.fast.example.come
node2.slow.example.com mpi=node2.fast.example.come
\end{lstlisting}

The remapped host names/addresses are listed in the ``\cmdarg{mpi}''
key.

Unless otherwise specified, this \file{lam-hostmap.txt} file is used
at run-time.  An alternate hostmap file can also be specified on the
\cmd{mpirun} command line using the \issiparam{mpi\_\-hostmap} SSI
parameter:

\lstset{style=lam-cmdline}
\begin{lstlisting}
shell$ mpirun C -ssi mpi_hostmap my_hostmap.txt my_mpi_application
\end{lstlisting}
% stupid emacs mode: $

This tells LAM to use the hostmap \file{my\_hostmap.txt} instead of
\file{\$sysconf/lam-hostmap.txt}.  The special filename
``\cmdarg{none}'' can also be used to indicate that no address
remapping should be performed.

\index{lam-hostmap.txt file@\file{lam-hostmap.txt} file|)}

\changeend{7.1}


%%%%%%%%%%%%%%%%%%%%%%%%%%%%%%%%%%%%%%%%%%%%%%%%%%%%%%%%%%%%%%%%%%%%%%%%%%%
%%%%%%%%%%%%%%%%%%%%%%%%%%%%%%%%%%%%%%%%%%%%%%%%%%%%%%%%%%%%%%%%%%%%%%%%%%%

\section{The LAM Test Suite}
\index{lamtests test suite@\file{lamtests} test suite}

The LAM test suite provides basic coverage of the LAM/MPI
installation.  If the test suite succeeds, users can be confident LAM
has been configured and installed properly.  The test suite is best
run with 4 or more nodes.  However, it can easily be run on two
machines.

The details of booting the LAM run-time environment are found in the
LAM/MPI User's Guide.  If you have never used LAM before, you may wish
to look at the quick overview chapter before running the LAM test
suite.  \file{boot\_schema}, below, is the name of a file containing a
list of hosts on which to run the test suite.

\index{lamtests test suite@\file{lamtests} test suite!lamd RPI
  warning@\rpi{lamd} RPI warning}

{\bf WARNING}: It is {\em strongly} discouraged to run the test suite
on the \rpi{lamd} RPI over a large number of nodes.  The \rpi{lamd}
RPI uses UDP for communication and does not scale well.  Running on a
large number of nodes can cause ``packet storm'' issues, and generally
degrade network performance for the duration of the test.  The default
configuration of the lamtests suite is to run on all available nodes
using all available RPI modules.  To run on a subset of available RPI
modules, specify the \cmdarg{MODES} parameter with a list of RPI
modules to use (not all versions of \cmd{make} allow this).  See the
example below.

\lstset{style=lam-shell}
\begin{lstlisting}
shell$ gunzip -c lamtests-7.1.2.tar.gz | tar xf -
shell$ cd lamtests-7.1.2
shell$ lamboot boot_schema
# ...lots of output...

shell$ ./configure
# ...lots of output...

shell$ make -k check MODES=''sysv usysv gm tcp''
# ...lots of output...

shell$ lamhalt
\end{lstlisting}

\index{lamtests test suite@\file{lamtests} test suite!common errors}
Common errors in the test suite include:

\begin{itemize}
\index{lamtests test suite@\file{lamtests} test suite!Fortran compiler errors}
\item Some Fortran compilers will confuse the GNU Autoconf fortran
  libraries test in the test suite's \cmd{configure} script (e.g.,
  some versions of the Intel Fortran compiler, and some versions of
  \cmd{g77} on OS X), causing the entire process to fail.  If the
  ``checking for dummy main to link with Fortran 77 libraries'' test
  fails, set the \envvar{FLIBS} environment variable to the output of
  the ``checking for Fortran 77 libraries'' test, but without the
  syntax errors that it incorrectly contains.  For example
  (artificially word-wrapped to fit on the page):

\lstset{style=lam-shell}
\begin{lstlisting}
checking for Fortran 77 libraries...  -L/u/jsquyres/local/lib
-llamf77mpi -lmpi -llam -lutil -lpthread'' -L\ -lpthread
-L/usr/local/intel/compiler70/ia32/lib -L/usr/lib -lintrins -lIEPCF90
-lF90 -limf -lm -lirc -lcxa -lunwind
checking for dummy main to link with Fortran 77 libraries... unknown
configure: error: linking to Fortran libraries from C fails
See `config.log' for more details.
\end{lstlisting}

  Note the extra quotes and empty ``{\tt -L}'' in the output.  Remove
  these syntax errors when setting the \envvar{FLIBS} variable:

\lstset{style=lam-cmdline}
\begin{lstlisting}
shell$ FLIBS=``-L/u/jsquyres/local/lib -llamf77mpi -lmpi -llam -lutil \
-lpthread -L/usr/local/intel/compiler70/ia32/lib -L/usr/lib -lintrins \
-lIEPCF90 -lF90 -limf -lm -lirc -lcxa -lunwind''
shell$ export FLIBS
shell$ ./configure ...
\end{lstlisting}
% stupid emacs mode: $

  \changebegin{7.1}
  
  In other cases, the test simply gets the wrong answer.  With
  \cmd{g77} on OS X, for example, the test may include
  ``\cmd{-lcrt2.o}'' in the argument list, which may cause linking
  problems such as duplicate symbols (which can be seen by looking in
  the \file{config.log} file).  In this case, setting \envvar{FLIBS}
  environment variable without the ``\cmd{-lcrt2.o}'' argument may fix
  the problem.

  \changeend{7.1}

  After setting \envvar{FLIBS} to this value, run \cmd{configure}
  again; the bad test will be skipped and \cmd{configure} will finish
  successfully.

\item If the test suite was not on a filesystem that is available on
  all nodes used in the testing, you may see a small number of errors
  during testing of the dynamic process features.  These errors are
  normal and can safely be ignored.

\index{lamtests test suite@\file{lamtests} test suite!BProc errors}
\item BProc systems will see errors in the \cmd{procname} test,
  complaining that the name returned by MPI was not what was
  expected.  This is normal and can safely be ignored.
  
\index{lamtests test suite@\file{lamtests} test suite!Myrinet errors}
\item If using the \rpi{gm} RPI on a Solaris system,\footnote{Solaris
    systems in particular are suseciptable to this problem because the
    native Myrinet message passing library (gm) is unable to ``pin''
    arbitrary memory under Solaris, and instead must rely on special
    DMA-malloc functions.} there may be limited amounts of DMA-able
  memory that LAM can access.  Some tests require $O(N)$ (or even
  $O(N^2)$) memory ($N$ is the number of processes in the test), and
  can exhaust the available DMA-able memory.  This is not a problem
  with LAM, but rather a problem with the DMA memory system.  Either
  tweak your operating system's settings to allow more DMA-able
  memory, or set the environment variable
  \ienvvar{LAM\_\-MPI\_\-test\_\-small\_\-mem} (to any value) before
  running the tests.  Setting this variable will cause some of the LAM
  tests to artificially decrease their message sizes such that most
  tests (if not all) should complete successfully.
  
\item Some tests require a fixed number of MPI processes (e.g., 6 or 8
  processes for \cmd{cart}, \cmd{graph}, \cmd{range}, \cmd{sub})
  rather than using the default LAM universe with ``C''.  This may
  cause problems with some RPI modules (such as \rpi{sysv}) that try
  to allocate resources on a per-process basis if the total number of
  processes causes oversubscription on the testing nodes (e.g.,
  running all 8 processes on a single node).  These errors are safe to
  ignore, or can be fixed by running on a larger number of nodes.
\end{itemize}

If there are any other errors, please see the troubleshooting section
(Section~\ref{sec:trouble}, page~\pageref{sec:trouble}).


