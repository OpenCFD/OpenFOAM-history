% -*- latex -*-
%
% Copyright (c) 2001-2004 The Trustees of Indiana University.  
%                         All rights reserved.
% Copyright (c) 1998-2001 University of Notre Dame. 
%                         All rights reserved.
% Copyright (c) 1994-1998 The Ohio State University.  
%                         All rights reserved.
% 
% This file is part of the LAM/MPI software package.  For license
% information, see the LICENSE file in the top level directory of the
% LAM/MPI source distribution.
%
% $Id: quick.tex,v 1.7 2003/06/20 14:36:32 jsquyres Exp $
%


\chapter{For the Impatient}
\label{sec:quick}

If you don't want to read the rest of the instructions, the following
should build a minimal installation of LAM/MPI in most situations
(note that some options will be disabled, particularly if supporting
libraries are installed in non-default locations, such as Myrinet,
BProc, Globus, PBS, etc.).

\lstset{style=lam-cmdline}
\begin{lstlisting}
shell$ gunzip -c lam-7.1.2.tar.gz | tar xf -
shell$ cd lam-7.1.2

# Set the desired C, C++, and Fortran compilers, unless using the GNU compilers is sufficient
shell$ ./configure CC=cc CXX=CC FC=f77 --prefix=/directory/to/install/in
# ...lots of output...

shell$ make
# ...lots of output...
shell$ make install
# ...lots of output...

# The following step is optional.  Ensure that $prefix/bin is in in your $path so that
# LAM's newly-created ``mpicc'' can be found before running this step.
shell$ make examples
# ...lots of output...
\end{lstlisting}

If you do not specify a prefix, LAM will first look for \icmd{lamclean} in
your path.  If \icmd{lamclean} is found, it will use the parent of the
directory where \icmd{lamclean} is located as the prefix.  Otherwise,
\file{/usr/local} is used (like most GNU software).

Now read Chapter~\ref{sec:releasenotes}
(page~\pageref{sec:releasenotes}); it contains information about the
new features of this release of LAM/MPI.

