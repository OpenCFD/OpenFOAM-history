% -*- latex -*-
%
% Copyright (c) 2001-2004 The Trustees of Indiana University.  
%                         All rights reserved.
% Copyright (c) 1998-2001 University of Notre Dame. 
%                         All rights reserved.
% Copyright (c) 1994-1998 The Ohio State University.  
%                         All rights reserved.
% 
% This file is part of the LAM/MPI software package.  For license
% information, see the LICENSE file in the top level directory of the
% LAM/MPI source distribution.
%
% $Id: ssi.tex,v 1.4 2003/10/15 17:17:56 adharurk Exp $
%

\chapter{System Services Interface (SSI) Overview}
\label{sec:ssi}
\index{SSI!overview|(}
\index{System Services Interface|see {SSI}}

The System Services Interface (SSI) makes up the core of LAM/MPI.  It
influences how many commands and MPI processes are executed.  This
chapter provides an overview of what SSI is and what users need to
know about how to use it to maximize performance of MPI applications.

%%%%%%%%%%%%%%%%%%%%%%%%%%%%%%%%%%%%%%%%%%%%%%%%%%%%%%%%%%%%%%%%%%%%%%%%%%%
%%%%%%%%%%%%%%%%%%%%%%%%%%%%%%%%%%%%%%%%%%%%%%%%%%%%%%%%%%%%%%%%%%%%%%%%%%%

\section{Types and Modules}
\index{SSI!module types}

SSI provides a component framework for the LAM run-time environment
(RTE) and the MPI communications layer.  Components are selected from
each type at run-time and used to effect the LAM RTE and MPI library.

There are currently four types of components used by
LAM/MPI:

\begin{itemize}
\item \kind{boot}: Starting the LAM run-time environment, used mainly
  with the \cmd{lamboot} command.

\item \kind{coll}: MPI collective communications, only used within MPI
  processes.

\item \kind{cr}: Checkpoint/restart functionality, used both within
  LAM commands and MPI processes.

\item \kind{rpi}: MPI point-to-point communications, only used within
  MPI processes.  
\end{itemize}

The LAM/MPI distribution includes instances of each component type
referred to as modules.  Each module is an implementation of the
component type which can be selected and used at run-time to provide
services to the LAM RTE and MPI communications layer.
Chapters~\ref{sec:lam-ssi} and~\ref{sec:mpi-ssi} list the modules that
are available in the LAM/MPI distribution.

%%%%%%%%%%%%%%%%%%%%%%%%%%%%%%%%%%%%%%%%%%%%%%%%%%%%%%%%%%%%%%%%%%%%%%%%%%%
%%%%%%%%%%%%%%%%%%%%%%%%%%%%%%%%%%%%%%%%%%%%%%%%%%%%%%%%%%%%%%%%%%%%%%%%%%%

\section{Terminology}

\begin{description}
\item[Available] The term ``available'' is used to describe a module
  that reports (at run-time) that it is able to run in the current
  environment.  For example, an RPI module may check to see if
  supporting network hardware is present before reporting that it is
  available or not.
  
  Chapters~\ref{sec:lam-ssi} and~\ref{sec:mpi-ssi} list the modules
  that are included in the LAM/MPI distribution, and detail the
  requirements for each of them to indicate whether they are available
  or not.
  
\item[Selected] The term ``selected'' means that a module has been
  chosen to be used at run-time.  Depending on the module type, zero
  or more modules may be selected.
  
\item[Scope] Each module selection has a scope depending on the type
  of the module.  ``Scope'' refers to the duration of the module's
  selection.  Table~\ref{tbl:ssi-module-scopes} lists the scopes for
  each module type.
\end{description}

\begin{table}[htbp]
  \centering
  \begin{tabular}{|l|p{4in}|}
    \hline
    \multicolumn{1}{|c|}{Type} &
    \multicolumn{1}{|c|}{Scope description} \\
    \hline
    \hline
    \kind{boot} & A module is selected at the beginning of
    \cmd{lamboot} (or \cmd{recon}) and is used for the duration of the
    LAM universe. \\
    \hline
    \kind{coll} & A module is selected every time an MPI communicator
    is created (including \mpiconst{MPI\_\-COMM\_\-WORLD} and
    \mpiconst{MPI\_\-COMM\_\-SELF}). It remains in use until that
    communicator has been freed. \\
    \hline
    \kind{cr} & Checkpoint/restart modules are selected at the
    beginning of an MPI job and remain in use until the job
    completes. \\
    \hline
    \kind{rpi} & RPI modules are selected during \mpifunc{MPI\_\-INIT}
    and remain in use until \mpifunc{MPI\_\-FINALIZE} returns. \\
    \hline
  \end{tabular}
  \caption{SSI module types and their corresponding scopes.}
  \label{tbl:ssi-module-scopes}
\end{table}

%%%%%%%%%%%%%%%%%%%%%%%%%%%%%%%%%%%%%%%%%%%%%%%%%%%%%%%%%%%%%%%%%%%%%%%%%%%
%%%%%%%%%%%%%%%%%%%%%%%%%%%%%%%%%%%%%%%%%%%%%%%%%%%%%%%%%%%%%%%%%%%%%%%%%%%

\section{SSI Parameters}
\label{sec:commands-ssi-module-parameters}
\index{SSI!parameter overview}

One of the founding principles of SSI is to allow the passing of
run-time parameters through the SSI framework.  This allows both the
selection of which modules will be used at run-time (by passing
parameters to the SSI framework itself) as well as tuning run-time
performance of individual modules (by passing parameters to each
module).
%
Although the specific usage of each SSI module parameter is defined by
either the framework or the module that it is passed to, the value of
most parameters will be resolved by the following:

\begin{enumerate}
\item If a valid value is provided via a run-time SSI parameter, use
  that.
  
\item Otherwise, attempt to calculate a meaningful value at run-time
  or use a compiled-in default value.\footnote{Note that many SSI
    modules provide configure flags to set compile-time defaults
    for ``tweakable'' parameters.
    See~\cite{lamteam03:_lam_mpi_install_guide}.}
\end{enumerate}

As such, it is typically possible to set a parameter's default value
when LAM is configured/compiled, but use a different value at run
time.

%%%%%%%%%%%%%%%%%%%%%%%%%%%%%%%%%%%%%%%%%%%%%%%%%%%%%%%%%%%%%%%%%%%%%%%%%%%

\subsection{Naming Conventions}

SSI parameter names are generally strings containing only letters and
underscores, and can typically be broken down into three parts.  For
example, the parameter \ssiparam{boot\_\-rsh\_\-agent} can be broken
into its three components:

\begin{itemize}
\item SSI module type: The first string of the name.  In this case, it
  is \ssiparam{boot}.

\item SSI module name: The second string of the name, corresponding to
  a specific SSI module.  In this case, it is \ssiparam{rsh}.
  
\item Parameter name: The last string in the name.  It may be an
  arbitrary string, and include multiple underscores.  In this case,
  it is \ssiparam{agent}.
\end{itemize}

Although the parameter name is technically only the last part of the
string, it is only proper to refer to it within its overall context.
Hence, it is correct to say ``the \ssiparam{boot\_\-rsh\_\-agent}
parameter'' as well as ``the \ssiparam{agent} parameter to the
\boot{rsh} boot module''.

Note that the reserved string \ssiparam{base} may appear as a module
name, referring to the fact that the parameter applies to all modules
of a give type.

%%%%%%%%%%%%%%%%%%%%%%%%%%%%%%%%%%%%%%%%%%%%%%%%%%%%%%%%%%%%%%%%%%%%%%%%%%%

\subsection{Setting Parameter Values}

SSI parameters each have a unique name and can take a single string
value.  The parameter/value pairs can be passed by multiple different
mechanisms.  Depending on the target module and the specific
parameter, mechanisms may include:

\begin{itemize}
\item Using command line flags when LAM was configured.
\item Setting environment variables before invoking LAM commands.
\item Using the \cmdarg{-ssi} command line switch to various LAM
  commands.
\item Setting attributes on MPI communicators.
\end{itemize}

Users are most likely to utilize the latter three methods.  Each is
described in detail, below.  Listings and explanations of available
SSI parameters are provided in Chapters~\ref{sec:lam-ssi}
and~\ref{sec:mpi-ssi} (pages~\pageref{sec:lam-ssi}
and~\pageref{sec:mpi-ssi}, respectively), categorized by SSI type and
module.

%%%%%

\subsubsection{Environment Variables}

SSI parameters can be passed via environment variables prefixed with
\envvar{LAM\_\-MPI\_\-SSI}.  For example, selecting which RPI module
to use in an MPI job can be accomplished by setting the environment
variable \envvar{LAM\_\-MPI\_\-SSI\_\-rpi} to a valid RPI module name
(e.g., \cmdarg{tcp}).

Note that environment variables must be set {\em before} invoking the
corresponding LAM/MPI commands that will use them.

%%%%%

\subsubsection{\cmdarg{-ssi} Command Line Switch}

LAM/MPI commands that interact with SSI modules accept the
\cmdarg{-ssi} command line switch.  This switch expects two parameters
to follow: the name of the SSI parameter and its corresponding value.
For example:

\lstset{style=lam-cmdline}
\begin{lstlisting}
shell$ mpirun C -ssi rpi tcp my_mpi_program
\end{lstlisting}
% stupid emacs mode: $

\noindent runs the \cmd{my\_\-mpi\_\-program} on all available CPUs in
the LAM universe using the \rpi{tcp} RPI module.

%%%%%

\subsubsection{Communicator Attributes}

Some SSI types accept SSI parameters via MPI communicator attributes
(notably the MPI collective communication modules).  These parameters
follow the same rules and restrictions as normal MPI attributes.  Note
that for portability between 32 and 64 bit systems, care should be
taken when setting and getting attribute values.  The following is an
example of portable attribute C code:

\lstset{style=lam-c}
\begin{lstlisting}
int flag, attribute_val;
void *set_attribute;
void **get_attribute;
MPI_Comm comm = MPI_COMM_WORLD;
int keyval = LAM_MPI_SSI_COLL_BASE_ASSOCIATIVE;

/* Set the value */
set_attribute = (void *) 1;
MPI_Comm_set_attr(comm, keyval, &set_attribute);

/* Get the value */
get_attribute = NULL;
MPI_Comm_get_attr(comm, keyval, &get_attribute, &flag);
if (flag == 1) {
  attribute_val = (int) *get_attribute;
  printf(``Got the attribute value: %d\n'', attribute_val);
}
\end{lstlisting}
% stupid emacs mode: $

Specifically, the following code is neither correct nor portable:

\lstset{style=lam-c}
\begin{lstlisting}
int flag, attribute_val;
MPI_Comm comm = MPI_COMM_WORLD;
int keyval = LAM_MPI_SSI_COLL_BASE_ASSOCIATIVE;

/* Set the value */
attribute_val = 1;
MPI_Comm_set_attr(comm, keyval, &attribute_val);

/* Get the value */
attribute_val = -1;
MPI_Comm_get_attr(comm, keyval, &attribute_val, &flag);
if (flag == 1)
  printf(``Got the attribute value: %d\n'', attribute_val);
\end{lstlisting}
% stupid emacs mode: $

\index{SSI!overview|)}

%%%%%%%%%%%%%%%%%%%%%%%%%%%%%%%%%%%%%%%%%%%%%%%%%%%%%%%%%%%%%%%%%%%%%%%%%%%
%%%%%%%%%%%%%%%%%%%%%%%%%%%%%%%%%%%%%%%%%%%%%%%%%%%%%%%%%%%%%%%%%%%%%%%%%%%

\section{Dynamic Shared Object (DSO) Modules}

\changebegin{7.1}

LAM has the capability of building SSI modules statically as part of
the MPI libraries or as dynamic shared objects (DSOs).  DSOs are
discovered and loaded into LAM processes at run-time.  This allows
adding (or removing) functionality from an existing LAM installation
without the need to recompile or re-link user applications.

The default location for DSO SSI modules is \file{\$prefix/lib/lam}.
If otherwise unspecified, this is where LAM will look for DSO SSI
modules.  However, the SSI parameter
\issiparam{base\_\-module\_\-path} can be used to specify a new
colon-delimited path to look for DSO SSI modules.  This allows users
to specify their own location for modules, if desired.

Note that specifying this parameter overrides the default location.
If users wish to augment their search path, they will need to include
the default location in the path specification.

\lstset{style=lam-cmdline}
\begin{lstlisting}
shell$ mpirun C -ssi base_module_path $prefix/lib/lam:$HOME/my_lam_modules ...
\end{lstlisting}
% stupid emacs mode: $

\changeend{7.1}

%%%%%%%%%%%%%%%%%%%%%%%%%%%%%%%%%%%%%%%%%%%%%%%%%%%%%%%%%%%%%%%%%%%%%%%%%%%
%%%%%%%%%%%%%%%%%%%%%%%%%%%%%%%%%%%%%%%%%%%%%%%%%%%%%%%%%%%%%%%%%%%%%%%%%%%

\section{Selecting Modules}

As implied by the previous sections, modules are selected at run-time
either by examining (in order) user-specified parameters, run-time
calculations, and compiled-in defaults.  The selection process
involves a flexible negotitation phase which can be both tweaked and
arbitrarily overriden by the user and system administrator.

%%%%%%%%%%%%%%%%%%%%%%%%%%%%%%%%%%%%%%%%%%%%%%%%%%%%%%%%%%%%%%%%%%%%%%%%%%%

\subsection{Specifying Modules}

Each SSI type has an implicit SSI parameter corresponding to the type
name indicating which module(s) to be considered for selection.  For
example, to specify in that the \rpi{tcp} RPI module should be used,
the SSI parameter \ssiparam{rpi} should be set to the value
\ssiparam{tcp}.  For example:

\lstset{style=lam-cmdline}
\begin{lstlisting}
shell$ mpirun C -ssi rpi tcp my_mpi_program
\end{lstlisting}
% stupid emacs mode: $

The same is true for the other SSI types (\kind{boot}, \kind{cr}, and
\kind{coll}), with the exception that the \kind{coll} type can be used
to specify a comma-separated list of modules to be considered as each
MPI communicator is created (including
\mpiconst{MPI\_\-COMM\_\-WORLD}).  For example:

\lstset{style=lam-cmdline}
\begin{lstlisting}
shell$ mpirun C -ssi coll smp,shmem,lam_basic my_mpi_program
\end{lstlisting}
% stupid emacs mode: $

\noindent indicates that the \coll{smp} and \coll{lam\_\-basic}
modules will potentially both be considered for selection for each MPI
communicator.

%%%%%%%%%%%%%%%%%%%%%%%%%%%%%%%%%%%%%%%%%%%%%%%%%%%%%%%%%%%%%%%%%%%%%%%%%%%

\subsection{Setting Priorities}

Although typically not useful to individual users, system
administrators may use priorities to set system-wide defaults that
influence the module selection process in LAM/MPI jobs.

Each module has an associated priority which plays role in whether a
module is selected or not.  Specifically, if one or more modules of a
given type are available for selection, the modules' priorities will
be at least one of the factors used to determine which module will
finally be selected.  Priorities are in the range $[-1, 100]$, with
$-1$ indicating that the module should not be considered for
selection, and $100$ being the highest priority.  Ties will be broken
arbitrarily by the SSI framework.

A module's priorty can be set run-time through the normal SSI
parameter mechanisms (i.e., environment variables or using the
\cmdarg{-ssi} parameter).  Every module has an implicit priority SSI
parameter in the form \ssiparam{$<$type$>$\_\-$<$module
  name$>$\_\-priority}.

For example, a system administrator may set environment variables in
system-wide shell setup files (e.g., \file{/etc/profile},
\file{/etc/bashrc}, or \file{/etc/csh.cshrc}) to change the default
priorities.  

%%%%%%%%%%%%%%%%%%%%%%%%%%%%%%%%%%%%%%%%%%%%%%%%%%%%%%%%%%%%%%%%%%%%%%%%%%%

\subsection{Selection Algorithm}

For each component type, the following general selection algorithm is
used:

\begin{itemize}
\item A list of all available modules is created.  If the user
  specified one or more modules for this type, only those modules are
  queried to see if they are available.  Otherwise, all modules are
  queried.
  
\item The module with the highest priority (and potentially meeting
  other selection criteria, depending on the module's type) will be
  selected.
\end{itemize}

Each SSI type may define its own additional selection rules.  For
example, the selection of \kind{coll}, \kind{cr}, and \kind{rpi}
modules may be inter-dependant, and depend on the supported MPI thread
level.  Chapter~\ref{sec:mpi-ssi} (page~\pageref{sec:mpi-ssi}) details
the selection algorithm for MPI SSI modules.

